\documentclass[a4paper,12pt]{article}

\usepackage[utf8]{inputenc}
\usepackage[czech]{babel}
\usepackage{amsmath, amssymb}
\usepackage{yhmath}
\usepackage{tikz}
\usepackage{tkz-euclide}
\usepackage{enumitem} % Pro lepší formátování seznamů

\begin{document}

\section*{Geometrický rozbor}

Úlohou je sestrojit střed $I$ kružnice vepsané trojúhelníku $ABC$, je-li dána kružnice opsaná $k$ (bez vyznačeného středu) a tři body $A, B, C$ na ní. K dispozici máme pouze pravítko s ryskou (umožňuje rýsovat přímky a kolmice).

\subsection*{1. Nalezení středu opsané kružnice ($O$)}
Abychom mohli pracovat s oblouky, potřebujeme nejprve najít střed kružnice~$k$. Využijeme vlastnosti Thaletovy věty a pravoúhelníku:
\begin{itemize}
    \item Vztyčíme-li v bodě $B$ kolmici na stranu $AB$, protne tato kolmice kružnici v bodě $K$. Podle Thaletovy věty je úsečka $AK$ průměrem kružnice.
    \item Analogicky vztyčíme kolmici v bodě $A$ na stranu $AB$, která protne kružnici v bodě $L$. Úsečka $BL$ je rovněž průměrem.
    \item Body $A, B, K, L$ tvoří vrcholy obdélníku vepsaného do kružnice. Průsečík úhlopříček $AK$ a $BL$ je hledaný střed $O$.
\end{itemize}

\subsection*{2. Sestrojení os úhlů}
Střed vepsané kružnice $I$ leží v průsečíku os vnitřních úhlů trojúhelníku.
\begin{itemize}
    \item Osa úhlu $\alpha$ (při vrcholu $A$) prochází středem $S_a$ oblouku $BC$ (toho oblouku, který neobsahuje bod $A$). (důkaz věty na konci)
    \item Bod $S_a$ leží na ose úsečky $BC$. Osa tětivy prochází středem kružnice $O$ a je kolmá k tětivě.
    \item Sestrojíme tedy kolmici z bodu $O$ na stranu $BC$. Její průsečík s kružnicí (v polorovině neobsahující $A$) je bod $S_a$. Polopřímka $AS_a$ je hledanou osou úhlu.
    \item Analogicky sestrojíme osu úhlu $\beta$ pomocí středu $S_b$ oblouku $AC$.
\end{itemize}

\section*{Konstrukční postup}

\begin{enumerate}[label=\textbf{\arabic*.}]
    \item Sestrojíme přímku $p \perp AB$ procházející bodem $B$.
    \item $K$; $K = p \cap k$ ($K \neq B$).
    \item Sestrojíme přímku $q \perp AB$ procházející bodem $A$.
    \item $L$; $L = q \cap k$ ($L \neq A$).
    \item $O$; $O = AK \cap BL$ (Střed opsané kružnice).
    \item Sestrojíme přímku $r$ procházející bodem $O$ tak, že $r \perp BC$.
    \item $S_a$; $S_a = r \cap k$, přičemž $S_a$ a $A$ leží v opačných polorovinách určených přímkou $BC$.
    \item Sestrojíme polopřímku $AS_a$ (osa úhlu $\alpha$).
    \item Sestrojíme přímku $s$ procházející bodem $O$ tak, že $s \perp AC$.
    \item $S_b$; $S_b = s \cap k$, přičemž $S_b$ a $B$ leží v opačných polorovinách určených přímkou $AC$.
    \item Sestrojíme polopřímku $BS_b$ (osa úhlu $\beta$).
    \item $I$; $I = \text{polopřímka } AS_a \cap \text{polopřímka } BS_b$.
\end{enumerate}

\vspace{2cm}

\begin{center}
% ZDE VLOŽÍME TVŮJ KÓD OBRÁZKU
\begin{tikzpicture}[scale=1.5]
  % --- ZADÁNÍ ---
  \tkzDefPoint(0,0){O}
  \tkzDefPoint(3,0){R} 
  \tkzDefPoint(160:3){A}
  \tkzDefPoint(210:3){B}
  \tkzDefPoint(300:3){C}
  
  \tkzDrawCircle(O,R)
  \tkzDrawPolygon[thick](A,B,C)
  \tkzLabelPoints[above left](A)
  \tkzLabelPoints[below left](B)
  \tkzLabelPoints[below right](C)

  % --- KROK 1: HLEDÁNÍ STŘEDU O (Obdélníková metoda) ---
  \tkzDefLine[orthogonal=through B](A,B) \tkzGetPoint{dir1}
  \tkzInterLC(B,dir1)(O,A) \tkzGetPoints{}{K}
  \tkzDrawLine[dashed, color=gray](B,K)
  \tkzDrawSegment[dashed, color=gray](A,K) 
  \tkzLabelPoints(K)
  
  \tkzDefLine[orthogonal=through A](A,B) \tkzGetPoint{dir2}
  \tkzInterLC(A,dir2)(O,A) \tkzGetPoints{L}{}
  \tkzDrawLine[dashed, color=gray](A,L)
  \tkzDrawSegment[dashed, color=gray](B,L) 
  \tkzLabelPoints(L)
  
  \tkzDrawPoint[color=red](O)
  \tkzLabelPoint[right, color=red](O){$O$}

  % --- KROK 2: OSY ÚHLŮ ---
  % A) Osa úhlu alfa
  \tkzDefLine[orthogonal=through O](B,C) \tkzGetPoint{dirBC}
  \tkzInterLC(O,dirBC)(O,A) \tkzGetPoints{S_a}{S_a_far} 
  \tkzDrawLine[dotted, color=blue](O,S_a)
  \tkzDrawPoint[color=blue](S_a)
  \tkzDrawLine[add=0 and 0.2, color=blue, thick](A,S_a)

  % B) Osa úhlu beta
  \tkzDefLine[orthogonal=through O](A,C) \tkzGetPoint{dirAC}
  \tkzInterLC(O,dirAC)(O,A) \tkzGetPoints{S_b_far}{S_b}
  \tkzDrawLine[dotted, color=green!60!black](O,S_b)
  \tkzDrawPoint[color=green!60!black](S_b)
  \tkzDrawLine[add=0 and 0.2, color=green!60!black, thick](B,S_b)

  % --- VÝSLEDEK ---
  \tkzInterLL(A,S_a)(B,S_b) \tkzGetPoint{I}
  \tkzDrawPoint[color=red, size=4](I)
  \tkzLabelPoint[left, color=red](I){$I$}
  
%   % Kontrolní vepsaná kružnice
%   \tkzDefPointBy[projection=onto A--B](I) \tkzGetPoint{T}
%   \tkzDrawCircle[color=red](I,T)
  % --- KONTROLA (Vepsaná kružnice) ---
  % Místo 'projection' použijeme průsečík kolmice
  
  % 1. Definujeme pomocný bod 'temp' na kolmici vedené z I k AB
  \tkzDefLine[orthogonal=through I](A,B) \tkzGetPoint{temp}
  
  % 2. Najdeme průsečík této kolmice se stranou AB -> to je bod dotyku T
  \tkzInterLL(I,temp)(A,B) \tkzGetPoint{T}
  
  % 3. Vykreslíme kružnici se středem I procházející bodem T
  \tkzDrawCircle[color=red](I,T)
\end{tikzpicture}
\end{center}

\paragraph{Důkaz tvrzení:}
Osa úhlu $\alpha$ (při vrcholu $A$) prochází středem $S_a$ oblouku $BC$. Důkaz vychází z věty o obvodových úhlech.

\begin{enumerate}
    \item \textbf{Rozdělení úhlu:}
    Protože polopřímka $A S_a$ je osou úhlu $\alpha$, dělí tento úhel na dvě poloviny. Platí tedy:
    \[
        \angle BAS_a = \angle CAS_a = \frac{\alpha}{2}
    \]

    \item \textbf{Vztah úhlů a oblouků:}
    Věta o obvodových úhlech říká, že velikost obvodového úhlu je přímo úměrná délce oblouku, nad kterým tento úhel leží.
    \begin{itemize}
        \item Úhel $\angle BAS_a$ přísluší oblouku $B S_a$.
        \item Úhel $\angle CAS_a$ přísluší oblouku $C S_a$.
    \end{itemize}

    \item \textbf{Porovnání:}
    Protože se rovnají úhly ($\frac{\alpha}{2} = \frac{\alpha}{2}$), musí se rovnat i délky oblouků, které jim odpovídají:
    \[
        \left| \overset{\frown}{BS_a} \right| = \left| \overset{\frown}{CS_a} \right|
    \]

    \item \textbf{Závěr:}
    Pokud jsou oblouky $B S_a$ a $C S_a$ stejně dlouhé a navazují na sebe v bodě $S_a$, pak bod $S_a$ nutně leží ve středu oblouku $BC$.
\end{enumerate}

\vspace{2cm}

\begin{center}
    \begin{tikzpicture}[scale=1.8]
        \tkzDefPoint(0,0){O}
  \tkzDefPoint(2,0){X}
  \tkzDrawCircle(O,X)
  
  % Body
  \tkzDefPoint(110:2){A}
  \tkzDefPoint(200:2){B}
  \tkzDefPoint(340:2){C}
  
  % Osa úhlu
  \tkzDefLine[bisector](B,A,C) \tkzGetPoint{bisector_dir}
  \tkzInterLC(A,bisector_dir)(O,X) \tkzGetPoints{junk}{Sa}
  
  % Vykreslení
  \tkzDrawPolygon(A,B,C)
  \tkzDrawSegment[color=red, thick](A,Sa)
  \tkzDrawSegments[dashed](B,Sa C,Sa)
  
  % Značky úhlů
  \tkzMarkAngle[size=0.6, mark=|](B,A,Sa)
  \tkzMarkAngle[size=0.6, mark=|](Sa,A,C)
  
  % Popisky
  \tkzLabelPoints[above left](A)
  \tkzLabelPoints[below left](B)
  \tkzLabelPoints[right](C)
  \tkzLabelPoint[below](Sa){$S_a$}
  
  % Textové vysvětlení do obrázku
  \tkzLabelAngle[pos=1.1](B,A,Sa){$\alpha/2$}
  \tkzLabelAngle[pos=1.1](Sa,A,C){$\alpha/2$}
  
  % Zvýraznění oblouků (jen schematicky barvou tětiv)
  \tkzDrawSegment[color=blue, ultra thick](B,Sa)
  \tkzDrawSegment[color=blue, ultra thick](C,Sa)
\end{tikzpicture}
\end{center}

\end{document}